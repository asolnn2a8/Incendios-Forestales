



% Introducción breve, solo una pequeña introducción a su problema (contexto), resumen de lo realizado en presentación 1 y 2 y lo hecho para el informe final más el cumplimiento de los objetivos que tenían.

\subsection{Contexto y Motivación}
En el último tiempo se ha visto una creciente en el número de incendios en todo el mundo. Cómo lo es en el caso de Chile, el Amazonas (Brasil) o Australia. Es por esto, que se planteó en utilizar las herramientas de Aprendizaje de Máquinas para poder ayudar a combatir y prevenir los incendios.

\subsection{Objetivos}
En este trabajo se intentará plantear mecanismos para predecir las causas de un incendio, dadas algunas condiciones (como la fecha, el lugar en dónde ocurrió el incendio o el tiempo empleado en controlar el incendio).

\subsection{Problema, preguntas realizadas e hipótesis}
Las preguntas se quieren responder son las siguientes:
\begin{itemize}
    \item Dada la hora, fecha, ubicación geográfica, área total del incendio y el tiempo necesario para extinguir un incendio. ¿Es posible predecir la causa del incendio?
    \item Dadas las condiciones anteriores. ¿Es posible saber si un incendio fue causado por error humano, por la naturaleza o con fines maliciosos?
\end{itemize}
